% $Log: abstract.tex,v $
% Revision 1.1  93/05/14  14:56:25  starflt
% Initial revision
% 
% Revision 1.1  90/05/04  10:41:01  lwvanels
% Initial revision
% 
%
%% The text of your abstract and nothing else (other than comments) goes here.
%% It will be single-spaced and the rest of the text that is supposed to go on
%% the abstract page will be generated by the abstractpage environment.  This
%% file should be \input (not \include 'd) from cover.tex.


\thispagestyle{plain}

\phantomsection

\addcontentsline{toc}{chapter}{Resumen}

\begin{center}
    \Large
    \textbf{Patrones de programación paralelos de alto nivel en arquitecturas de memoria distribuida}
    
    \vspace{0.4cm}
    \large
    Trabajo Fin de Máster
    
    \vspace{0.4cm}
    \textbf{Javier Prieto Cepeda}
    
    \vspace{0.9cm}
    \textbf{Resumen}
\end{center}

En los últimos años, los grandes volúmenes de flujo datos y los requisitos casi en tiempo real de las aplicaciones de transmisión de datos han incrementado la necesidad de nuevos algoritmos escalables e interfaces de programación para plataformas de memoria compartida y distribuida. Para contribuir en esta dirección, este trabajo presenta un nuevo back-end \acrshort{mpi} distribuido para \acrshort{grppi}, una interfaz genérica de alto nivel de C ++ de patrones paralelos de procesamiento intensivo de datos y streaming. Este back-end, como una nueva política de ejecución, admite la ejecución paralela distribuida e híbrida (distribución y memoria compartida) de los patrones de pipeline y farm, donde el modo híbrido combina la política \acrshort{mpi} con una memoria compartida \acrshort{grppi}. %, es decir, OpenMP, C ++ Threads o Intel TBB.
% Para abordar plataformas heterogéneas, hemos agregado el operador \ texttt {run \ _with} que actúa como un contenedor para patrones de transmisión y puede reemplazar la política de ejecución predeterminada por una más adecuada para una etapa \ pipeline específica.
Un análisis detallado de la política de ejecución de \acrshort{grppi} \acrshort{mpi} muestra considerables beneficios desde los puntos de vista de programación, flexibilidad y legibilidad. La evaluación experimental en una aplicación de streaming con diferentes escenarios de memoria compartida y distribuida indica ganancias de rendimiento considerables con respecto a las versiones secuenciales a expensas de gastos indirectos insignificantes de \acrshort{grppi}.


\vspace{0.7cm}

\textbf{Palabras clave: Patrones de programación paralelos, Procesamiento de streaming, Patrones de programación distribuidos, Programación C++, Programación genérica}

\afterpage{\blankpage} % blank page