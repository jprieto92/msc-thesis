% -*-latex-*-
% 
% For questions, comments, concerns or complaints:
% thesis@mit.edu
% 
%
% $Log: cover.tex,v $
% Revision 1.8  2008/05/13 15:02:15  jdreed
% Degree month is June, not May.  Added note about prevdegrees.
% Arthur Smith's title updated
%
% Revision 1.7  2001/02/08 18:53:16  boojum
% changed some \newpages to \cleardoublepages
%
% Revision 1.6  1999/10/21 14:49:31  boojum
% changed comment referring to documentstyle
%
% Revision 1.5  1999/10/21 14:39:04  boojum
% *** empty log message ***
%
% Revision 1.4  1997/04/18  17:54:10  othomas
% added page numbers on abstract and cover, and made 1 abstract
% page the default rather than 2.  (anne hunter tells me this
% is the new institute standard.)
%
% Revision 1.4  1997/04/18  17:54:10  othomas
% added page numbers on abstract and cover, and made 1 abstract
% page the default rather than 2.  (anne hunter tells me this
% is the new institute standard.)
%
% Revision 1.3  93/05/17  17:06:29  starflt
% Added acknowledgements section (suggested by tompalka)
% 
% Revision 1.2  92/04/22  13:13:13  epeisach
% Fixes for 1991 course 6 requirements
% Phrase "and to grant others the right to do so" has been added to 
% permission clause
% Second copy of abstract is not counted as separate pages so numbering works
% out
% 
% Revision 1.1  92/04/22  13:08:20  epeisach

% NOTE:
% These templates make an effort to conform to the MIT Thesis specifications,
% however the specifications can change.  We recommend that you verify the
% layout of your title page with your thesis advisor and/or the MIT 
% Libraries before printing your final copy.

\title{A Complete Simulator for Volunteer Computing Environments}

\author{Sa\'ul Alonso Monsalve}
% If you wish to list your previous degrees on the cover page, use the 
% previous degrees command:
%       \prevdegrees{A.A., Harvard University (1985)}
% You can use the \\ command to list multiple previous degrees
%       \prevdegrees{B.S., University of California (1978) \\
%                    S.M., Massachusetts Institute of Technology (1981)}
%\department{Department of Electrical Engineering and Computer Science}

% If the thesis is for two degrees simultaneously, list them both
% separated by \and like this:
% \degree{Doctor of Philosophy \and Master of Science}
%\degree{Bachelor of Science in Computer Science and Engineering}

% As of the 2007-08 academic year, valid degree months are September, 
% February, or June.  The default is June.
%\degreemonth{June}
%\degreeyear{1990}
%\thesisdate{May 18, 1990}

%% By default, the thesis will be copyrighted to MIT.  If you need to copyright
%% the thesis to yourself, just specify the `vi' documentclass option.  If for
%% some reason you want to exactly specify the copyright notice text, you can
%% use the \copyrightnoticetext command.  
%\copyrightnoticetext{\copyright IBM, 1990.  Do not open till Xmas.}

% If there is more than one supervisor, use the \supervisor command
% once for each.
\supervisor{Félix García Carballeira}{Full Professor}

% This is the department committee chairman, not the thesis committee
% chairman.  You should replace this with your Department's Committee
% Chairman.
\chairman{Arthur C. Smith}{Chairman, Department Committee on Graduate Theses}

% Make the titlepage based on the above information.  If you need
% something special and can't use the standard form, you can specify
% the exact text of the titlepage yourself.  Put it in a titlepage
% environment and leave blank lines where you want vertical space.
% The spaces will be adjusted to fill the entire page.  The dotted
% lines for the signatures are made with the \signature command.
%\maketitle

% The abstractpage environment sets up everything on the page except
% the text itself.  The title and other header material are put at the
% top of the page, and the supervisors are listed at the bottom.  A
% new page is begun both before and after.  Of course, an abstract may
% be more than one page itself.  If you need more control over the
% format of the page, you can use the abstract environment, which puts
% the word "Abstract" at the beginning and single spaces its text.

%% You can either \input (*not* \include) your abstract file, or you can put
%% the text of the abstract directly between the \begin{abstractpage} and
%% \end{abstractpage} commands.

% First copy: start a new page, and save the page number.
\afterpage{\blankpage} % blank page
\clearpage

\thispagestyle{empty}
\vspace*{\fill} 
\begin{quote}
\epigraph{\large \textit{``No hay nada más bonito, que poder dedicar tu tiempo a tu vocación. Lucha por tus sueños, y trabaja en lo que te haga vibrar.''}}{\large \flushright \textbf{Florin Isaila, UC3M.}}
\end{quote}
\vspace*{\fill} 

%\begin{comment}
\chapter*{Agradecimientos}
\addcontentsline{toc}{chapter}{\textit{Agradecimientos}}%

Con este trabajo pongo fin a una etapa muy bonita de mi vida, en la cual no solamente me llevo todo lo que he aprendido en la universidad, sino también grandes experiencias y el placer de haber conocido a grandísimas personas.

En primer lugar quiero agradecer a mis padres, Tomás y Raquel, y a mi hermano, David, la paciencia que han tenido conmigo y los ánimos que me han dado. ¡Que sería de mi sin vosotros! También quiero agradecer a mis abuelos su preocupación y ánimos en estos años, ¡Por fin dejo de daros la brasa con mis agobios! A mis tías, mis tíos, y mis pequeñajos, que siempre me sacan una sonrisa.

Por otro lado, también quiero agradecer su paciencia, ánimos y los grandes momentos que me ha brindado a Miriam. Cariño, contigo empecé mi aventura en la universidad, y contigo la he finalizado. Gracias por todo, te quiero.

También me gustaría agradecer a mis tutores de este trabajo, Javier y Manuel su esfuerzo y dedicación en estos meses. Pese a que la universidad no me ha permitido añadir a Manuel (bendita burocracia...), unas palabras en un documento administrativo no quitan su trabajo y ayuda durante meses. Nuevamente, os doy las gracias Manu y Javi.

Por último, pero no por ello menos importante, me gustaría destacar a esas personas que han hecho que mi día a día en la universidad haya sido más llevadero y, por qué no, una alegría en muchas ocasiones. Mis compañeros en ARCOS, Carlos, Fran, Estefanía, Silvina, David, Javi y Alberto, gracias por tantas risas en el laboratorio (y fuera de él). También a Álex, ¡Cuanto me has enseñado, amigo!. A los chicos del lab (Óscar, Jaime y Roberto) que siempre te arrancan una sonrisa. Y a Saúl, Guille, Álvaro, Rubén y Mario; con quienes he compartido grandísimos momentos y desahogos. A todos vosotros, Gracias.

%\end{comment}

\thispagestyle{empty}

\afterpage{\blankpage} % blank page
\clearpage

%%%%%%%%%%%%%%%%%%%%%%%%%%%%%%%%%%%%%%%%%%%%%%%%%%%%%%%%%%%%%%%%%%%%%%
% -*-latex-*-


% Uncomment the next line if you do NOT want a page number on your
% abstract and acknowledgments pages.
% \pagestyle{empty}
%\setcounter{savepage}{\thepage}

% $Log: abstract.tex,v $
% Revision 1.1  93/05/14  14:56:25  starflt
% Initial revision
% 
% Revision 1.1  90/05/04  10:41:01  lwvanels
% Initial revision
% 
%
%% The text of your abstract and nothing else (other than comments) goes here.
%% It will be single-spaced and the rest of the text that is supposed to go on
%% the abstract page will be generated by the abstractpage environment.  This
%% file should be \input (not \include 'd) from cover.tex.


\thispagestyle{plain}

\phantomsection

\addcontentsline{toc}{chapter}{Resumen}

\begin{center}
    \Large
    \textbf{Patrones de programación paralelos de alto nivel en arquitecturas de memoria distribuida}
    
    \vspace{0.4cm}
    \large
    Trabajo Fin de Máster
    
    \vspace{0.4cm}
    \textbf{Javier Prieto Cepeda}
    
    \vspace{0.9cm}
    \textbf{Resumen}
\end{center}

En los últimos años, los grandes volúmenes de flujo datos y los requisitos casi en tiempo real de las aplicaciones de transmisión de datos han incrementado la necesidad de nuevos algoritmos escalables e interfaces de programación para plataformas de memoria compartida y distribuida. Para contribuir en esta dirección, este trabajo presenta un nuevo back-end \acrshort{mpi} distribuido para \acrshort{grppi}, una interfaz genérica de alto nivel de C ++ de patrones paralelos de procesamiento intensivo de datos y streaming. Este back-end, como una nueva política de ejecución, admite la ejecución paralela distribuida e híbrida (distribución y memoria compartida) de los patrones de pipeline y farm, donde el modo híbrido combina la política \acrshort{mpi} con una memoria compartida \acrshort{grppi}. %, es decir, OpenMP, C ++ Threads o Intel TBB.
% Para abordar plataformas heterogéneas, hemos agregado el operador \ texttt {run \ _with} que actúa como un contenedor para patrones de transmisión y puede reemplazar la política de ejecución predeterminada por una más adecuada para una etapa \ pipeline específica.
Un análisis detallado de la política de ejecución de \acrshort{grppi} \acrshort{mpi} muestra considerables beneficios desde los puntos de vista de programación, flexibilidad y legibilidad. La evaluación experimental en una aplicación de streaming con diferentes escenarios de memoria compartida y distribuida indica ganancias de rendimiento considerables con respecto a las versiones secuenciales a expensas de gastos indirectos insignificantes de \acrshort{grppi}.


\vspace{0.7cm}

\textbf{Palabras clave: Patrones de programación paralelos, Procesamiento de streaming, Patrones de programación distribuidos, Programación C++, Programación genérica}

\afterpage{\blankpage} % blank page

\clearpage

% $Log: abstract.tex,v $
% Revision 1.1  93/05/14  14:56:25  starflt
% Initial revision
% 
% Revision 1.1  90/05/04  10:41:01  lwvanels
% Initial revision
% 
%
%% The text of your abstract and nothing else (other than comments) goes here.
%% It will be single-spaced and the rest of the text that is supposed to go on
%% the abstract page will be generated by the abstractpage environment.  This
%% file should be \input (not \include 'd) from cover.tex.

\thispagestyle{plain}

\phantomsection

\addcontentsline{toc}{chapter}{Abstract}

\begin{center}
    \Large
    \textbf{High-Level parallel programming patterns for distributed memory platforms}
    
    
    \vspace{0.4cm}
    \large
    Master's Thesis
    
    \vspace{0.4cm}
    \textbf{Javier Prieto Cepeda}
    
    \vspace{0.9cm}
    \textbf{Abstract}
\end{center}

In the recent years, the large volumes of stream data and the near real-time requirements of data streaming applications have exacerbated the need for new scalable algorithms and programming interfaces for distributed and shared-memory platforms. To contribute in this direction, this work presents a new distributed \acrshort{mpi} back end for \acrshort{grppi}, a C ++ high-level generic interface of data-intensive and stream processing parallel patterns. This back end, as a new execution policy, supports the distributed and hybrid (distributed and shared-memory) parallel execution of the pipeline and farm patterns, where the hybrid mode combines the \acrshort{mpi} policy with a \acrshort{grppi} shared-memory one. %, i.e., OpenMP, C++ Threads or Intel TBB. 
%To tackle heterogeneous platforms, we have added the operator \texttt{run\_with} that acts as a wrapper for streaming patterns and is able to replace the default execution policy by a more suitable one for a specific \pipeline stage. 
A detailed analysis of the \acrshort{grppi} \acrshort{mpi} execution policy reports considerable benefits from the programmability, flexibility and readability points of view. The experimental evaluation on a streaming application with different distributed and shared-memory scenarios reports considerable performance gains with respect to the sequential versions at the expense of negligible \acrshort{grppi} overheads.


\vspace{0.7cm}

\textbf{Keywords: Parallel Patterns, Stream Processing, Distributed Patterns, C++ Programming, Generic Programming} 

\afterpage{\blankpage} % blank page
\clearpage
